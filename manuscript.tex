\PassOptionsToPackage{unicode=true}{hyperref} % options for packages loaded elsewhere
\PassOptionsToPackage{hyphens}{url}
%
\documentclass[english,,man]{apa6}
\usepackage{lmodern}
\usepackage{amssymb,amsmath}
\usepackage{ifxetex,ifluatex}
\usepackage{fixltx2e} % provides \textsubscript
\ifnum 0\ifxetex 1\fi\ifluatex 1\fi=0 % if pdftex
  \usepackage[T1]{fontenc}
  \usepackage[utf8]{inputenc}
  \usepackage{textcomp} % provides euro and other symbols
\else % if luatex or xelatex
  \usepackage{unicode-math}
  \defaultfontfeatures{Ligatures=TeX,Scale=MatchLowercase}
\fi
% use upquote if available, for straight quotes in verbatim environments
\IfFileExists{upquote.sty}{\usepackage{upquote}}{}
% use microtype if available
\IfFileExists{microtype.sty}{%
\usepackage[]{microtype}
\UseMicrotypeSet[protrusion]{basicmath} % disable protrusion for tt fonts
}{}
\IfFileExists{parskip.sty}{%
\usepackage{parskip}
}{% else
\setlength{\parindent}{0pt}
\setlength{\parskip}{6pt plus 2pt minus 1pt}
}
\usepackage{hyperref}
\hypersetup{
            pdftitle={Non-word repetition in Yélî Dnye},
            pdfkeywords={phonology, non-word repetition, development},
            pdfborder={0 0 0},
            breaklinks=true}
\urlstyle{same}  % don't use monospace font for urls
\usepackage{graphicx,grffile}
\makeatletter
\def\maxwidth{\ifdim\Gin@nat@width>\linewidth\linewidth\else\Gin@nat@width\fi}
\def\maxheight{\ifdim\Gin@nat@height>\textheight\textheight\else\Gin@nat@height\fi}
\makeatother
% Scale images if necessary, so that they will not overflow the page
% margins by default, and it is still possible to overwrite the defaults
% using explicit options in \includegraphics[width, height, ...]{}
\setkeys{Gin}{width=\maxwidth,height=\maxheight,keepaspectratio}
\setlength{\emergencystretch}{3em}  % prevent overfull lines
\providecommand{\tightlist}{%
  \setlength{\itemsep}{0pt}\setlength{\parskip}{0pt}}
\setcounter{secnumdepth}{0}

% set default figure placement to htbp
\makeatletter
\def\fps@figure{htbp}
\makeatother

% Manuscript styling
\usepackage{upgreek}
\captionsetup{font=singlespacing,justification=justified}

% Table formatting
\usepackage{longtable}
\usepackage{lscape}
% \usepackage[counterclockwise]{rotating}   % Landscape page setup for large tables
\usepackage{multirow}		% Table styling
\usepackage{tabularx}		% Control Column width
\usepackage[flushleft]{threeparttable}	% Allows for three part tables with a specified notes section
\usepackage{threeparttablex}            % Lets threeparttable work with longtable

% Create new environments so endfloat can handle them
% \newenvironment{ltable}
%   {\begin{landscape}\begin{center}\begin{threeparttable}}
%   {\end{threeparttable}\end{center}\end{landscape}}
\newenvironment{lltable}{\begin{landscape}\begin{center}\begin{ThreePartTable}}{\end{ThreePartTable}\end{center}\end{landscape}}

% Enables adjusting longtable caption width to table width
% Solution found at http://golatex.de/longtable-mit-caption-so-breit-wie-die-tabelle-t15767.html
\makeatletter
\newcommand\LastLTentrywidth{1em}
\newlength\longtablewidth
\setlength{\longtablewidth}{1in}
\newcommand{\getlongtablewidth}{\begingroup \ifcsname LT@\roman{LT@tables}\endcsname \global\longtablewidth=0pt \renewcommand{\LT@entry}[2]{\global\advance\longtablewidth by ##2\relax\gdef\LastLTentrywidth{##2}}\@nameuse{LT@\roman{LT@tables}} \fi \endgroup}

% \setlength{\parindent}{0.5in}
% \setlength{\parskip}{0pt plus 0pt minus 0pt}

% \usepackage{etoolbox}
\makeatletter
\patchcmd{\HyOrg@maketitle}
  {\section{\normalfont\normalsize\abstractname}}
  {\section*{\normalfont\normalsize\abstractname}}
  {}{\typeout{Failed to patch abstract.}}
\makeatother
\shorttitle{NWR in Yélî Dnye}
\author{Alejandrina Cristia\textsuperscript{1}\ \& Marisa Casillas\textsuperscript{2}}
\affiliation{
\vspace{0.5cm}
\textsuperscript{1} Laboratoire de Sciences Cognitives et de Psycholinguistique, Département d'Etudes cognitives, ENS, EHESS, CNRS, PSL University\\\textsuperscript{2} Max Planck Institute for Psycholinguistics}
\authornote{All data are made available in a repository in the Open Science Framework. AC acknowledges the support of the Agence Nationale de la Recherche (ANR-17-CE28-0007 LangAge, ANR-16-DATA-0004, ANR-14-CE30-0003, ANR-17-EURE-0017); and the J. S. McDonnell Foundation Understanding Human Cognition Scholar Award. 
 


Correspondence concerning this article should be addressed to Alejandrina Cristia, 29, rue d’Ulm, 75005 Paris, France. E-mail: alecristia@gmail.com}
\keywords{phonology, non-word repetition, development\newline\indent Word count: ~xxx words}
\DeclareDelayedFloatFlavor{ThreePartTable}{table}
\DeclareDelayedFloatFlavor{lltable}{table}
\DeclareDelayedFloatFlavor*{longtable}{table}
\makeatletter
\renewcommand{\efloat@iwrite}[1]{\immediate\expandafter\protected@write\csname efloat@post#1\endcsname{}}
\makeatother
\usepackage{lineno}

\linenumbers
\usepackage{csquotes}
\ifnum 0\ifxetex 1\fi\ifluatex 1\fi=0 % if pdftex
  \usepackage[shorthands=off,main=english]{babel}
\else
  % load polyglossia as late as possible as it *could* call bidi if RTL lang (e.g. Hebrew or Arabic)
  \usepackage{polyglossia}
  \setmainlanguage[]{english}
\fi

\title{Non-word repetition in Yélî Dnye}

\date{}

\abstract{
ADD
}

\begin{document}
\maketitle

\hypertarget{todo}{%
\subsection{TODO}\label{todo}}

\begin{itemize}
\tightlist
\item
  fix yaml
\item
  add ana based on phonemes \& expand on levenshtein
\item
  add ana based on number of characters
\item
  add summary of types of error: \%semantic, \%deletion (which sounds?) \%subst (which are the most common sounds)
\item
  maybe add to the last ana the prevalence (ie out of all the gh which proportion get transformed)
\item
  add analysis on the basis of demo chars
\item
  discussion
\end{itemize}

\hypertarget{introduction}{%
\subsection{Introduction}\label{introduction}}

Although infants begin to learn about their native language's phonology within the first year, many studies suggest that in perception and production, in phonetics and phonology, their knowledge continues to develop throughout childhood (e.g., Hazan \& Barrett, 2000). One common task in this line of research is nonword repetition (NWR). In NWR studies, participants are presented auditorily with an item that is phonologically legal but lexically meaningless in the language children are learning. The child should immediately try to say it back without changing anything. Accuracy is thought to reflect long-term phonological knowledge (which allows the child to perceive the item accurately even though it is not a real word they have encountered before) as well as online phonological working memory (to encode the item in the interval between hearing it and saying it back) and flexible productiong patterns (to produce the item accurately even though it had never been pronounced before).

NWR has been used to seek answers to a variety of theoretical questions, including what are the links between phonology, working memory, and the lexicon (Bowey, 2001), as well as for applications, notably using nwr as a diagnostic for language delays and disorders (Estes, Evans, \& Else-Quest, 2007). Since non-words can be generated in many languages, it has been used acros a wide array of languages, particularly in Europe (Meir, Walters, \& Armon-Lotem, 2016,@is08042009language).

In this study, we report on NWR results among children learning Yêly Dnyé, an isolate spoken in Rossel Island, PNG, with an unusually dense phonological inventory. In our implementation of NWR, we made sure that some of the items contained typologically rare and/or challenging sounds. MISSING: WHY DO WE DO THIS? WHAT IS THE POINT?

\hypertarget{intro-to-the-language-mc---please-feel-free-to-throw-away-anything-that-is-not-useful}{%
\subsubsection{\texorpdfstring{Intro to the language (({\textbf{???}})) - please feel free to throw away anything that is not useful!}{Intro to the language ((???)) - please feel free to throw away anything that is not useful!}}\label{intro-to-the-language-mc---please-feel-free-to-throw-away-anything-that-is-not-useful}}

\begin{itemize}
\tightlist
\item
  complexity in the vowel system
\item
  complexity in the consonant system
\item
  word shapes
\item
  typical word length
\item
  although not the focus of this paper, high use of suppletion in verbal paradigms, other features of language, see Levinson XXX for details
\end{itemize}

\hypertarget{intro-to-the-people-mc---please-feel-free-to-throw-away-anything-that-is-not-useful}{%
\subsubsection{\texorpdfstring{Intro to the people (({\textbf{???}})) - please feel free to throw away anything that is not useful!}{Intro to the people ((???)) - please feel free to throw away anything that is not useful!}}\label{intro-to-the-people-mc---please-feel-free-to-throw-away-anything-that-is-not-useful}}

Little is known about language development in children growing up in Rossel Island, a community of primarily subsistence farmers who tend to reside in close-knitted villages where child care is distributed across many individuals, and who typically speak Yélî Dnyé, a phonologically and lexically complex language.

\begin{itemize}
\tightlist
\item
  usually monolingual at home
\item
  schooling in English but it starts at age XX, so not relevant here
\item
  however, some use of English due to immigrants \& children of immigrants
\item
  children spend a lot of time with other children
\item
  most parents are subsistence farmers
\item
  parental education generally varies between XX and YY
\end{itemize}

\hypertarget{brief-review-of-nwr-for-our-purposes}{%
\subsubsection{Brief review of NWR for our purposes}\label{brief-review-of-nwr-for-our-purposes}}

There is some variation in the presentation procedure and structure of items found in previous NWR work. For example, items are often presented orally by the experimenter (Torrington Eaton, Newman, Ratner, \& Rowe, 2015), although an increasing number of studies have turned to playing back the stimuli in order to have greater control of the stability of the presentation (Brandeker \& Thordardottir, 2015). Additionally, while some studies have used 10-15 non-words, others have employed up to 46 unique items (Piazzalunga, Previtali, Pozzoli, Scarponi, \& Schindler, 2019). Often, authors modulate structural complexity, typically measured in terms of item length (measured in number of syllables) and/or syllable structure (open as opposed to closed syllables, Gallon, Harris, \& Van der Lely, 2007).

Previous work seems to avoid difficult sounds, but we felt this was important to represent Yélî Dnye, so we also varied this factor. We designed a relatively large number of items but, aware that this may render the task longer and more tiresome, we split some of the items across children. This allowed us to get information about repetition accuracy of more items.

Naturally, designing the task in this way may render the study of individual variaiton within the population more difficult because different children are exposed to different items. However, a review of previous work on individual variation suggested to us that many individual differences effects are relatively small, and would not be detectable with the sample size that we could collect in a given visit.

That said, we contribute to the literature by also reporting descriptive analyses of individual variation that could potentially be integrated in meta-analytic efforts. Based on previous work, we looked at potential improvements with age (Farmani et al., 2018; Kalnak, Peyrard-Janvid, Forssberg, \& Sahlén, 2014; Vance, Stackhouse, \& Wells, 2005), and potential negative effects of bilingual exposure (Brandeker \& Thordardottir, 2015; Meir \& Armon-Lotem, 2017; Meir et al., 2016). Previous work typically finds no significant differences as a function of maternal education (e.g., Farmani et al., 2018; Kalnak et al., 2014; Meir \& Armon-Lotem, 2017) or child gender (Chiat \& Roy, 2007). Although previous research has not often investigated potential effects of birth order on NWR, there is a sizable literature on these effects in other language tasks (Havron et al., 2019), and therefore we report on those too.

\hypertarget{research-questions}{%
\subsubsection{Research questions}\label{research-questions}}

We sought to address the following questions:

\begin{itemize}
\tightlist
\item
  What is the overall repetition score, according to a variety of relevant metrics (whole word, phoneme based, distance)?
\item
  How does score change as a function of item complexity (number of syllablex, sound complexity)?
\item
  How frequent are errors that result in real words? Is that a function of item complexity?
\item
  What are some common patterns of phonological errors?
\item
  Is variation attributable to child age, sex, birth order, monolingual status, and/or parental education?
\end{itemize}

In view of the hypothesis-driven nature of this work, we considered boosting the interpretational value of this evidence by announcing our analysis plans prior to conducting them. However, we realized that even pre-registering an analysis would be equivocal because we do not have enough power to look at all relationships of interest, and often to detect any of the known effects. For instance, an effect of stimulus length and age has been reported as significant sometimes, but Tables XX and XX show that even these effects can be quite small (data from Cristia, Farabolini, Scaff, Havron, \& Stieglitz, 2020). \textbf{add table with r/ds from previous work})
Moreover, by virtue of this being a language-specific study, it is unclear that our method (including saliently the items we used) is comparable in precision to previous NWR studies. Therefore, all analyses here are descriptive and should be considered exploratory.

\begin{figure}
\centering
\includegraphics{manuscript_files/figure-latex/prevlit-fig-1.pdf}
\caption{\label{fig:prevlit-fig}Proportion of non-words correctly repeated as a function of age (in years), study (first author and year regardless of the number of authors), and, when available, shorter versus longer non-words.}
\end{figure}

\hypertarget{methods}{%
\subsection{Methods}\label{methods}}

\hypertarget{stimuli}{%
\subsubsection{Stimuli}\label{stimuli}}

Many NWR studies are based on a fixed list of 12-16 items that vary in length between 1 and 4 syllables, often additionally varying syllable complexity and/or cluster presence and complexity, always meeting the condition that they do not mean anything in the target language (e.g., Balladares, Marshall, \& Griffiths, 2016; Wilsenach, 2013). We kept the same variation in item length and the non-meaningfulness requirement, but we did not vary syllable complexity and clusters because these are vanishingly rare in Yély Dnye. We also increased the number of items an individual child would be tested on, so that a child would get up to 23 items to repeat (note that up to other work has also used 24-30 items: Jaber-Awida, 2018; Kalnak et al., 2014), and we created more items and distributed them across children, so as to increase the coverage, and be able to study more items.

A first list of candidate items was generated in 2018 by selecting simple consonants (\enquote{p}, \enquote{t}, \enquote{d}, \enquote{k}, \enquote{m}, \enquote{n}, \enquote{w}, \enquote{y}) and vowels (\enquote{i}, \enquote{o},\enquote{u}, \enquote{a},\enquote{e}) that were combined into consonant-vowel syllables, further sampling the space of 1- to 4-syllable sequences. These candidates were automatically checked against Levinson's 2015 dictionary and removed from consideration if they appeared in the dictionary. The second author presented them orally to three local research assistants, who were asked to repeat them and further say whether they were real words. Any item for which two or more of the assistants reported them having a meaning or some form of association was excluded.

A second list of candidate items was generated in 2019 by selecting complex consonants and systematically crossing them with all the vowels in the Yélî inventory to produce consonant-vowel monosyllables. As before, items were automatically excluded if they appeared in the dictionary. Additionally, since hearing vowel length in monosyllables in isolation is challenging, any item that had a short/long real word neighbor was filtered out. Since the phonology and phonetics of Yélî is still in the process of being described (CITE {\textbf{???}} please fill in), there could have been undocumented constraints that rendered items illegal. Therefore, we made sure that the precise consonant-vowel sequence occurred in some real word in the dictionary (i.e., that there was a longer word included the monosyllable as a subsequence). These candidates were presented to one informant, for a final check that they did not mean anything. Together with the 2018 selection, they were recorded using a headset \textbf{XXX ({\textbf{???}}) please fill in} and an Olympus \textbf{XXX ({\textbf{???}}) please fill in} from the written form presented together with the same item orally (by the second author). The complete recorded list was finally presented to two more informants, who could repeat all the items and who confirmed there were no real words. Even so, there was one monosyllable that was often identified as a real word (intended \enquote{yî} /yXX/; identified as \enquote{yi} /yi/, \emph{tree}). This item is removed from analyses.

\textbf{({\textbf{???}}) can you please add the IPA in // for all of the items in the table?}
The final list is composed of three practice items; 20 monoysllables containing sounds that are less frequent in the world's languages than singleton plosives; 8 bisyllables; 12 trisyllables; and 4 quadrisyllables (see Table 2).

Practice
wî, poni, nopimade

Mono
dpa
dp:a
dpâ
dpéé
dpê
dpi
dpu
kp:ââ
kpu
tp:a
tpâ
tpê
gh:ââ
ghuu
lva
lv:ê
lvi
t:êê
);

(
kamo
kipo
kani
tupa
noki
piwa
towi
nomi);

(
nademo
meyadi
diyeto
widone
nuyedi
tumowe
pedumi
dimope
tiwune
wumipo
nayeki
mituye);

(
todiwuma
wadikeno
nomiwake
diponate).

A Praat script was written to randomize this list 20 times, and split it into two sublists, to generate 40 different elicitation sets. The 40 elicitation sets are available online from \url{https://osf.io/5qspb}. The split had the following constraints:

\begin{itemize}
\tightlist
\item
  the same three items were selected as practice items and used in all 40 elicitation sets
\item
  splits were done within each length group from the 2018 items (i.e., separately for 2, 3, and 4-syllable items); and among onset groups for the difficult monosyllables generated in 2019 (i.e., all the monosyllables starting with tp were split into 2 sublists). Since some of these groups had an odd number of items, one of the sublists was slightly longer than the other (20 versus 23).
\item
  once the sublist split had been done, items were randomized such that all children heard first the 3 practice items in a fixed order (1, 2, and 4 syllables), a randomized version of their sublist selection of difficult onset items, and randomized versions of their 2-syllables, then 3-syllables, and finally 4-syllable items.
\end{itemize}

\hypertarget{procedure}{%
\subsubsection{Procedure}\label{procedure}}

We tried to balance three desiderata: That children would not be unduly exposed to the items before they themselves had to repeat them; that children would feel comfortable doing this task with us; and that the community would feel safe with us doing this task with their children. Moreover, there were also some logistic constraints in terms of the space availability. As a result, the places where elicitation took place varied across the hamlets.

We visited four different hamlets once, and attempted to test all eligible children present at the time, to prevent the items \enquote{spreading} through hearsay. In the first hamlet, we tested children in five different places, with some children being tested inside a house and others tested on the veranda. The complete list of places and the ways in which they met the desiderata mentioned above can be found in the raw data, available from online supplementary materials.

The child was donned a headset (\textbf{xx ({\textbf{???}}) please fill in} for most of the children, SHURE WH20 XLR headset with a dynamic microphone for the rest), recorded into the left channel into a Tascam DR40x digital recorder. For most children, the headset could not stay comfortably on the child's head, and thus it was placed on the child's shoulders, with the microphone carefully placed close to the child's mouth. A local informant sat next to the child, to would provide the instructions and, if needed, coach the child to make sure, using the three practice items as well as real words, that they understood that the task was to repeat the items precisely without changing anything. An experimenter (the first author) delivered the elicitation stimuli to the local informant and the child over headphones.

The first phase was making sure the child understood the task. This was explained orally and the first training item was presented. Often, children froze and did not say anything. If this happened, then we followed this procedure. First, the informant insisted. If the child still did not say anything, the informant asked the child to repeat a real word, and another, and another. If the child could repeat these correctly, then we provided the recorded training item over headphones again. Most children successfully started repeating the items presented over headphones at this point; a few further needed the local informant to model the behavior (i.e., they would hear the item again, and she would say it; then we would play it again, and ask the child to say it). A small minority still failed to repeat the item after hearing it over headphones. If that occurred, we tried with the second training item, at which point some children got it and could continue. A small minority, however, failed to repeat this one, as well as the third training item, in which case we stopped the test altogether.

NWR studies vary in whether children are provided with several opportunities to hear and say the item. To have a fixed and clear procedure, we decided that items other than the inital three training ones would not be repeated unless the child made an attempt to produce them. If this attempt was judged correct by the local informant, then the experimenter would move on to the next item (whispering this over a separate headset that was recording onto the right track of the same Tascam). If the local informant heard a deviation, she indicated to the experimenter that the item needed to be repeated, and up to 5 attempts were allowed.

Whenever siblings from the same family were tested, an attempt was made to test first the older and then the younger child, and always on different elicitaiton sets.

\hypertarget{coding}{%
\subsubsection{Coding}\label{coding}}

The local informant asked for the item to be repeated when she thought the repetition was not correct. We will call this \emph{online scoring}, and it reflects a wholesale impression of whether the item was correctly or incorrectly repeated.

We sought additional information by asking a local informant to listen through each child production, paired with the auditory target the child had been provided with, and make a judgment of whether the item was correctly or incorrectly repeated. We additionally asked her to transcribe exactly what the child said, providing some examples of the types of errors children in general make (without making specific reference to Yélî sounds or the items in the elicitation sets). This \emph{offline scoring} provides both accuracy and qualitative desciptions of what the child said that are likely more reliable than those of the non-native coder.

\hypertarget{analyses}{%
\subsubsection{Analyses}\label{analyses}}

For NWR accuracy, we considered separately the child's first and subsequent attempts. These were scored as correct or incorrect on the fly depending on whether the native experimenter asked for the item to be repeated; as well as offline. Some NWR studies employ phoneme-based accuracy in addition to or instead of word-level accuracy (e.g., Cristia et al., 2020). We calculated accuracy as the number of phonemes that could be aligned across the target and attempt, divided by the number of phonemes of whichever item was longer (the target or the attempt). Although previous work does not use distance metrics, we additionally report those.

Finally, for describing children's patterns of errors, all repetitions of a given target were taken into account. We describe the proportion of items where the change resulted in a real word (semantic errors); and classify the most common phonological errors.

\hypertarget{participants}{%
\subsubsection{Participants}\label{participants}}

This study was approved as part of a larger research effort by \textbf{({\textbf{???}}) add info from Middy's FAIR reply}. Participation was voluntary, with children being invited to come and participate. Regardless of how they performed, children were provided with a snack as compensation. Children who came up to participate but then refused were nonetheless provided with the snack.

A total 55 children were tested, from 34 of families, in five hamlets. Some children could not be included for the following reasons: refused participation or failed to repeat items presented over headphones even after coaching (N=0), spoke too softly to allow offline coding (N=5). In addition, 2 teenagers were tested to put younger children at ease; their data is not included in analyses below. The remaining 40 children (14 girls) were aged 6.96 years (range 3.92-11.03 years). There were 32 children exposed only to Yélî in the home, 6 children who were also exposed to another language in the home, and , 2 for whom this information was missing. Maternal years of education averaged 8.24 years (range 6-12 years; 2 children had this information missing).\footnote{Education is often reported in even years because people typically complete two-year cycles.} In terms of birth order, 0 were first borns, 5 second, 4 third, 2 forth, 6 fifth, 5 sixth, and 1 did not have this information.

\hypertarget{results}{%
\subsection{Results}\label{results}}

\hypertarget{preliminary-analyses}{%
\subsubsection{Preliminary analyses}\label{preliminary-analyses}}

\begin{figure}
\centering
\includegraphics{manuscript_files/figure-latex/Fig1-first_vs_others-1.pdf}
\caption{(\#fig:Fig1-first\_vs\_others)NWR scores for individual participants averaging separately their first attempts and all other attempts.}
\end{figure}

We first checked whether accuracy varies between first and subsequent presentations of an item by averaging word-level accuracy at the participant level separately for first attempts and subsequent repetitions. As shown in Figure 1, participants' mean word-level accuracy became more heterogeneous in subsequent repetitions. Surprisingly, subsequent repetitions (M = 42, SD = 28)
were on average less accurate than first ones (M = 64, SD = 15), t(36) = 5.32, p = 0. Given the uncertainty in whether previous work used only the first or all repetitions, and since behavior degraded and became more heterogeneous in subsequent repetitions, the rest of the analyses focus on only the first repetitions.

\hypertarget{overall-repetition-accuracy}{%
\subsubsection{Overall repetition accuracy}\label{overall-repetition-accuracy}}

Taking into account only the first attempts, the overall repetition accuracy measured in whole words is 72\% based on the online coding (i.e., whether the native experimenter asked for the target to be re-played), and 65\% when the offline coding is considered. On average, the phoneme-based normalized Levenshtein distance was 22\%, meaning that about 20\% of phonemes were substituted, inserted, or deleted.

ADD GRAPH WITH PHONEME BASED IN Y AND WORD BASED IN X

\hypertarget{accuracy-a-function-of-item-complexity}{%
\subsubsection{Accuracy a function of item complexity}\label{accuracy-a-function-of-item-complexity}}

We then inspected whether accuracy varied as a function of word length.

\begin{enumerate}
\def\labelenumi{\arabic{enumi}.}
\item
  Children are more accurate for mono-syllables than longer items
\item
  The length distribution in Yélî words is more balanced than that in English, and thus the performance decline for poly- versus mono-syllables may be less pronounced than that for English. \textbf{Check for work on European languages that may have looked into this}
\item
  Similarly, we do not know of NWR research that manipulates the difficulty of the sounds that are included in the items, but word naming and other research suggests that children are more accurate when producing easy and/or typologically common sounds than difficult and/or typologically rare sounds {[}CITE{]}. Therefore, we expect higher accuracy for items with common sounds than in those with rare sounds.
\item
  The Yélî sound inventory is very large and compressed, with many similar sounds that are acoustic and articulatory neighbors. Therefore, this may constitute a pressure for children to have finer auditory skills (and perhaps more precise articulations) than children speaking languages with a simpler inventory. As a result, differences between easier and harder items may be smaller in this work than in other research. \textbf{no work looking at consonants \& vowels? no work looking at nasal vowels in particular?}
\end{enumerate}

\textbf{(MC: but we can try and do a cursory analysis based on the corpora we have from Steve and my transcription of naturalistic interactions!)}

\hypertarget{patterns-of-errors}{%
\subsubsection{Patterns of errors}\label{patterns-of-errors}}

\begin{itemize}
\tightlist
\item
  How frequent are errors that result in real words? Is that a function of item complexity? What are some common patterns of phonological errors?
\end{itemize}

\hypertarget{factor-structuring-individual-variation}{%
\subsubsection{Factor structuring individual variation}\label{factor-structuring-individual-variation}}

explainable by child age, sex, birth order, monolingual status, and/or parental education?
3. Children's accuracy increases with child age.
4. Non-monolingual Yélî children are less accurate than monolingual ones when tested on the society-dominant language (we did not test any non-dominant language)
5. As revious NWR evidence on this is mixed, but general findings on language development suggest that children whose mothers are more educated are more accurate than children whose mothers are less educated.
6. To our knowledge, there is no previous NWR work on this, but other research suggests that first-born children should outperform later-born children

\begin{enumerate}
\def\labelenumi{\arabic{enumi}.}
\setcounter{enumi}{3}
\tightlist
\item
  Anecdotally Yélî children grow up in close-knitted communities and thus may receive significant portions of their language input from people not in their nuclear family
  (or at least from people other than their mothers, who tend to be the non-native speakers). If so, the difference between monolinguals and not monolinguals may be smaller than that found in other work
  . That said, one recent study on the same population shows that most child-directed input in the first 2 years does come from the mother
  , so in so far as this input has a crucial formational role, then there may still be a performance gap between these two groups.
\item
  In the Rossel community, formal education plays an extremely minor role in ensuring individual's success, is not a good index of relative socio-economic status, and furthermore there is only a narrow range of variation in maternal educational attainment. This may lead to no or only very small advantages for children whose mothers are more educated, provided that the causal chain between maternal education and child language is via SES more broadly. However, if education directly boosts maternal verbal skills and the incidence of verbal behavior (as suggested by CITE), then we should still see a difference along this factor.
\item
  One main causal path between birth order and language development is via parental input (CITE). Given our arguments above for how mothers may not be as important among Rossel people than in other places, then the performance gap between first borns and later borns may be smaller.
\end{enumerate}

\hypertarget{discussion}{%
\subsection{Discussion}\label{discussion}}

\begin{itemize}
\tightlist
\item
  What is the overall repetition accuracy (whole word, phoneme based, distance)?
\item
  How does this change as a function of item complexity (number of syllables, sound complexity)?
\item
  How frequent are errors that result in real words? Is that a function of item complexity?
\item
  Is individual variation explainable by child age, sex, birth order, monolingual status, and/or parental education?
\end{itemize}

\newpage

\hypertarget{acknowledgments}{%
\subsection{Acknowledgments}\label{acknowledgments}}

We are grateful to informants and individuals who participated in the study. AC acknowledges financial and institutional support from Agence Nationale de la Recherche (ANR-17-CE28-0007 LangAge, ANR-16-DATA-0004 ACLEW, ANR-14-CE30-0003 MechELex, ANR-17-EURE-0017) and the J. S. McDonnell Foundation Understanding Human Cognition Scholar Award. MC blabla NWO Veni Innovational Scheme Grant (XXX-XX-XXX).

\hypertarget{references}{%
\section{References}\label{references}}

\setlength{\parindent}{-0.5in}
\setlength{\leftskip}{0.5in}

\hypertarget{refs}{}
\leavevmode\hypertarget{ref-balladares2016socio}{}%
Balladares, J., Marshall, C., \& Griffiths, Y. (2016). Socio-economic status affects sentence repetition, but not non-word repetition, in Chilean preschoolers. \emph{First Language}, \emph{36}(3), 338--351. \url{https://doi.org/10.1177/0142723715626067}

\leavevmode\hypertarget{ref-bowey2001nonword}{}%
Bowey, J. A. (2001). Nonword repetition and young children's receptive vocabulary: A longitudinal study. \emph{Applied Psycholinguistics}, \emph{22}(3), 441--469.

\leavevmode\hypertarget{ref-brandeker2015language}{}%
Brandeker, M., \& Thordardottir, E. (2015). Language exposure in bilingual toddlers: Performance on nonword repetition and lexical tasks. \emph{American Journal of Speech-Language Pathology}, \emph{24}(2), 126--138.

\leavevmode\hypertarget{ref-chiat2007preschool}{}%
Chiat, S., \& Roy, P. (2007). The preschool repetition test: An evaluation of performance in typically developing and clinically referred children. \emph{Journal of Speech, Language, and Hearing Research}, \emph{50}(2), 429--443.

\leavevmode\hypertarget{ref-cristia2020infant}{}%
Cristia, A., Farabolini, G., Scaff, C., Havron, N., \& Stieglitz, J. (2020). Infant-directed input and literacy effects on phonological processing: Non-word repetition scores among the tsimane'. \emph{Preprint}.

\leavevmode\hypertarget{ref-estes2007differences}{}%
Estes, K. G., Evans, J. L., \& Else-Quest, N. M. (2007). Differences in the nonword repetition performance of children with and without specific language impairment: A meta-analysis. \emph{Journal of Speech, Language, and Hearing Research}, \emph{50}(1), 177--195.

\leavevmode\hypertarget{ref-farmani2018normalization}{}%
Farmani, H., Sayyahi, F., Soleymani, Z., Labbaf, F. Z., Talebi, E., \& Shourvazi, Z. (2018). Normalization of the non-word repetition test in farsi-speaking children. \emph{Journal of Modern Rehabilitation}, \emph{12}(4), 217--224.

\leavevmode\hypertarget{ref-gallon2007non}{}%
Gallon, N., Harris, J., \& Van der Lely, H. (2007). Non-word repetition: An investigation of phonological complexity in children with grammatical sli. \emph{Clinical Linguistics \& Phonetics}, \emph{21}(6), 435--455.

\leavevmode\hypertarget{ref-havron2019effect}{}%
Havron, N., Ramus, F., Heude, B., Forhan, A., Cristia, A., Peyre, H., \& Group, E. M.-C. C. S. (2019). The effect of older siblings on language development as a function of age difference and sex. \emph{Psychological Science}, \emph{30}(9), 1333--1343.

\leavevmode\hypertarget{ref-hazan2000development}{}%
Hazan, V., \& Barrett, S. (2000). The development of phonemic categorization in children aged 6--12. \emph{Journal of Phonetics}, \emph{28}(4), 377--396.

\leavevmode\hypertarget{ref-is08042009language}{}%
IS0804, C. A. (2009). Language impairment in a multilingual society: Linguistic patterns and the road to assessment. \emph{Brussels: COST Office. Available Online at: Http://Www. Bi-Sli. Org}.

\leavevmode\hypertarget{ref-jabere2018xperiment}{}%
Jaber-Awida, A. (2018). Experiment in non word repetition by monolingual Arabic preschoolers. \emph{Athens Journal of Philology}, \emph{5}(4), 317--334. \url{https://doi.org/10.30958/ajp.5-4-4}

\leavevmode\hypertarget{ref-kalnak2014nonword}{}%
Kalnak, N., Peyrard-Janvid, M., Forssberg, H., \& Sahlén, B. (2014). Nonword repetition--a clinical marker for specific language impairment in swedish associated with parents' language-related problems. \emph{PloS One}, \emph{9}(2), e89544.

\leavevmode\hypertarget{ref-meir2017independent}{}%
Meir, N., \& Armon-Lotem, S. (2017). Independent and combined effects of socioeconomic status (ses) and bilingualism on children's vocabulary and verbal short-term memory. \emph{Frontiers in Psychology}, \emph{8}, 1442.

\leavevmode\hypertarget{ref-meir2016disentangling}{}%
Meir, N., Walters, J., \& Armon-Lotem, S. (2016). Disentangling sli and bilingualism using sentence repetition tasks: The impact of l1 and l2 properties. \emph{International Journal of Bilingualism}, \emph{20}(4), 421--452.

\leavevmode\hypertarget{ref-piazzalunga2019articulatory}{}%
Piazzalunga, S., Previtali, L., Pozzoli, R., Scarponi, L., \& Schindler, A. (2019). An articulatory-based disyllabic and trisyllabic non-word repetition test: Reliability and validity in italian 3-to 7-year-old children. \emph{Clinical Linguistics \& Phonetics}, \emph{33}(5), 437--456.

\leavevmode\hypertarget{ref-torrington2015non}{}%
Torrington Eaton, C., Newman, R. S., Ratner, N. B., \& Rowe, M. L. (2015). Non-word repetition in 2-year-olds: Replication of an adapted paradigm and a useful methodological extension. \emph{Clinical Linguistics \& Phonetics}, \emph{29}(7), 523--535.

\leavevmode\hypertarget{ref-vance2005speech}{}%
Vance, M., Stackhouse, J., \& Wells, B. (2005). Speech-production skills in children aged 3--7 years. \emph{International Journal of Language \& Communication Disorders}, \emph{40}(1), 29--48.

\leavevmode\hypertarget{ref-wilsenach2013phonological}{}%
Wilsenach, C. (2013). Phonological skills as predictor of reading success: An investigation of emergent bilingual Northern Sotho/English learners. \emph{Per Linguam: a Journal of Language Learning= Per Linguam: Tydskrif vir Taalaanleer}, \emph{29}(2), 17--32. \url{https://doi.org/10.5785/29-2-554}

\end{document}
